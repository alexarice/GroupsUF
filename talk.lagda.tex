\documentclass[aspectratio=169,presentation]{beamer}

\usepackage{agda}
% \usepackage{fancyvrb}
\usepackage{amsthm}

\newtheorem{prop}[theorem]{Proposition}
\usepackage{amsmath}
\usepackage{microtype}
% \DefineVerbatimEnvironment{code}{Verbatim}{}

\usepackage[utf8]{inputenc}

\usepackage{newunicodechar}
\newunicodechar{λ}{\ensuremath{\mathnormal\lambda}}
\newunicodechar{ℓ}{\ensuremath{\mathnormal\ell}}
\newunicodechar{𝓖}{\ensuremath{\mathcal{G}}}
\newunicodechar{≡}{\ensuremath{\equiv}}
\newunicodechar{⟨}{\ensuremath{\langle}}
\newunicodechar{⟩}{\ensuremath{\rangle}}
\newunicodechar{⌊}{\ensuremath{\lfloor}}
\newunicodechar{⌋}{\ensuremath{\rfloor}}
\newunicodechar{→}{\ensuremath{\mathnormal\to}}
\newunicodechar{∀}{\ensuremath{\mathnormal\forall}}
\newunicodechar{∎}{\ensuremath{\mathnormal\qed}}
\newunicodechar{∘}{\ensuremath{\mathnormal\circ}}
\newunicodechar{⁻}{\ensuremath{\mathnormal{}^{-}}}
\newunicodechar{¹}{\ensuremath{\mathnormal{}^{1}}}
\newunicodechar{≈}{\ensuremath{\mathnormal\approx}}
\newunicodechar{′}{\ensuremath{\mathnormal{{}^\prime}}}
\newunicodechar{ε}{\ensuremath{\varepsilon}}
\newunicodechar{η}{\ensuremath{\eta}}
\newunicodechar{Σ}{\ensuremath{\Sigma}}
\newunicodechar{∈}{\ensuremath{\in}}
\newunicodechar{ι}{\ensuremath{\iota}}

\usetheme{Warsaw}

% \affiliation{?}

\title{Strictly Associative Group Theory using Univalence}
\author{Alex Rice\inst{1}}
\institute[1]{University of Cambridge}
\date{HoTT/UF 2023}
\titlegraphic{\includegraphics[width=0.3\textwidth]{../cambridge}}
\setbeamertemplate{footline}[frame number]
\setbeamertemplate{navigation symbols}{}

\begin{document}

{
  \setbeamertemplate{footline}{}
  \setbeamertemplate{headline}{} %
  \begin{frame}[noframenumbering]
\maketitle
\end{frame}}
\begin{frame}{Outline}
  \tableofcontents
\end{frame}

\section{What did I do?}

\begin{frame}
  \frametitle{Motivation}
  \begin{code}[hide]
    module talk where
    open import Cubical.Algebra.Group.Base
    open import Cubical.Algebra.Group.Properties
    open import Cubical.Foundations.Structure
    open import Cubical.Foundations.Prelude
    open import Groups.Symmetric.Representable using (RSymGroup) renaming (inc≡ to ι≡)

    variable
      A B C D : Type
  \end{code}

  \begin{code}
    InvUniqueLeft : ∀ {ℓ} (𝓖 : Group ℓ) → Type ℓ
    InvUniqueLeft 𝓖 = ∀ g h → h · g ≡ 1g → h ≡ inv g
      where
      open GroupStr (𝓖 .snd)
  \end{code}
  \only<2>{\begin{code}
    inv-unique-left : ∀ {ℓ} (𝓖 : Group ℓ) → InvUniqueLeft 𝓖
    inv-unique-left 𝓖 g h p =
      h                ≡⟨ sym (·IdR h) ⟩
      h · 1g           ≡⟨ cong (h ·_) (sym (·InvR g)) ⟩
      h · (g · inv g)  ≡⟨ ·Assoc h g (inv g) ⟩
      (h · g) · inv g  ≡⟨ cong (_· inv g) p ⟩
      1g · inv g       ≡⟨ ·IdL (inv g) ⟩
      inv g ∎
      where
          open GroupStr (𝓖 .snd)
  \end{code}}

  \only<3>{\begin{code}
    inv-unique-left-strict : ∀ {ℓ} (𝓖 : Group ℓ) → InvUniqueLeft 𝓖
    inv-unique-left-strict 𝓖 = strictify InvUniqueLeft
      λ g h p →
        h · 1g         ≡⟨ cong (h ·_) (sym (·InvR g))  ⟩
        h · g · inv g  ≡⟨ cong (_· inv g) p            ⟩
        1g · inv g ∎
        where
          open GroupStr (RSymGroup 𝓖 .snd)
          open import Groups.Reasoning 𝓖 using (strictify)
  \end{code}}

\end{frame}

\begin{frame}
  \frametitle{Strictify}
  \begin{itemize}
  \item Given a group \AgdaBound{𝓖}, we create a new group \AgdaFunction{RSymGroup}\AgdaSpace{}\AgdaBound{𝓖}.
    \begin{theorem}[Cayley's Theorem]
      Every group is isomorphic to a subgroup of a symmetric group.
    \end{theorem}
  \item In \AgdaFunction{RSymGroup}\AgdaSpace{}\AgdaBound{𝓖}, various rules hold by reflexivity.
  \item We show that \AgdaFunction{RSymGroup}\AgdaSpace{}\AgdaBound{𝓖} is isomorphic to 𝓖.
  \item By univalence and the structure identity principle, \AgdaFunction{RSymGroup}\AgdaSpace{}\AgdaBound{𝓖} is equal to 𝓖.
  \item The \AgdaFunction{strictify} function transports a proof from \AgdaFunction{RSymGroup}\AgdaSpace{}\AgdaBound{𝓖} back to 𝓖.
  \end{itemize}
\end{frame}


\begin{frame}
   In the strictified group the following equations hold definitionally:
  \begin{itemize}
  \item \(a  (b c) = (a b) c\),
  \item \(a 1 = a = 1 a\),
  \item \({a^{-1}}^{-1} = a\),
  \item and \((f g)^{-1} = g^{-1} \cdot f^{-1}\).
  \end{itemize}
\end{frame}

\section{How did I do it?}

\begin{frame}
  \frametitle{Functions compose strictly}
  \begin{theorem}[Cayley's Theorem]
    Every group is isomorphic to a subgroup of a symmetric group.
  \end{theorem}

  \pause{}
  \begin{code}[hide]
    module A where
  \end{code}
  \begin{code}
      _∘_ : (f : B → C) → (g : A → B) → (A → C)
      (f ∘ g) x = f (g x)

      comp-assoc  :  (f : C → D)
                  →  (g : B → C)
                  →  (h : A → B)
                  →  f ∘ (g ∘ h) ≡ (f ∘ g) ∘ h
      comp-assoc f g h = refl
  \end{code}
\end{frame}

\begin{frame}
  \frametitle{Do invertible functions compose strictly?}
  \begin{code}[hide]
    module X where
  \end{code}
  \begin{code}
      record Inverse (A : Type) (B : Type) : Type where
        field
          ↑ : A → B
          ↓ : B → A
          ε : ∀ x  → ↓ (↑ x) ≡ x
          η : ∀ y  → ↑ (↓ y) ≡ y
  \end{code}
\end{frame}

\begin{frame}
  \frametitle{Strict invertible functions}
  \begin{code}[hide]
    module Y where
  \end{code}
  \begin{code}
      record Inverse (A : Type) (B : Type) : Type where
        constructor ⌊_,_,_,_⌋
        field
          ↑ : A → B
          ↓ : B → A
          ε : ∀ b {x}  → x ≡ ↓ b  → ↑ x ≡ b
          η : ∀ a {y}  → y ≡ ↑ a  → ↓ y ≡ a
  \end{code}
  \begin{code}[hide]
      infixr 30 _∘_
  \end{code}
  \begin{code}
      _∘_ : Inverse B C → Inverse A B → Inverse A C
      _∘_ ⌊ f , g , p , q ⌋ ⌊ f' , g' , p' , q' ⌋ =
        ⌊  (λ x → f (f' x)) ,
           (λ y → g' (g y)) ,
           (λ b r → p b (p' (g b) r)) ,
           (λ a r → q' a (q (f' a) r)) ⌋

  \end{code}
  \begin{code}[hide]
      open Inverse public
  \end{code}
\end{frame}

\begin{frame}
  \frametitle{Strict invertible functions}
  \begin{code}
      assoc :  (f : Inverse C D)
            →  (g : Inverse B C)
            →  (h : Inverse A B)
            →  f ∘ (g ∘ h) ≡ (f ∘ g) ∘ h
      assoc f g h = refl

      id-inv : Inverse A A
      id-inv = ⌊  (λ x → x) , (λ x → x) ,
                  (λ b r → r) , (λ a r → r) ⌋

      id-unit-left :  (f : Inverse A B)
                   →  id-inv ∘ f ≡ f
      id-unit-left f = refl

      id-unit-right :  (f : Inverse A B)
                    →  f ∘ id-inv ≡ f
      id-unit-right f = refl
  \end{code}
\end{frame}

\begin{frame}
  \frametitle{Strict invertible functions}
  \begin{code}
      inv-inv : Inverse A B → Inverse B A
      inv-inv ⌊ f , g , ε , η ⌋ = ⌊ g , f , η , ε ⌋

      inv-involution :  (f : Inverse A B)
                     →  inv-inv (inv-inv f) ≡ f
      inv-involution f = refl

      inv-comp :  (f : Inverse B C)
               →  (g : Inverse A B)
               →  inv-inv (f ∘ g) ≡ inv-inv g ∘ inv-inv f
      inv-comp f g = refl
  \end{code}
\end{frame}

\begin{frame}
  \frametitle{Representable functions}
  The map \(\iota : g \mapsto g \cdot \_\) includes the group 𝓖 in the symmetric group. We now want to restrict the symmetric group to those functions that are in the image of \(\iota\).

  \begin{prop}
    A function \(f : \mathcal{G} \to \mathcal{G}\) is in the image of \(\iota\) if and only if for all \(g,h \in 𝓖\), \(f (g \cdot h) = f (g) \cdot h\).
  \end{prop}

  \pause

  \begin{code}[hide]
      module _ (𝓖 : Group ℓ-zero) where
        open GroupStr (𝓖 .snd)
  \end{code}
  \begin{code}
        Representable : Inverse ⟨ 𝓖 ⟩ ⟨ 𝓖 ⟩ → Type
        Representable f = ∀ x g h → x ≡ g · h → ↑ f x ≡ ↑ f g · h

        Repr : Type
        Repr = Σ[ f ∈ Inverse ⟨ 𝓖 ⟩ ⟨ 𝓖 ⟩ ] Representable f
  \end{code}
\end{frame}

\begin{frame}
  \frametitle{Representable symmetric group}

  \begin{itemize}
  \item<+-> Let \AgdaFunction{RSymGroup}\AgdaSpace{}\AgdaBound{𝓖} be the subgroup of the symmetric group on \(\mathcal{G}\) consisting of those functions that are representable.

  \item<+-> This subgroup still has strict composition.
  \item<+-> The inclusion \(ι\) is an isomorphism from \(\mathcal{G}\) to the representable symmetric group.
  \item<+-> By univalence we get an equality:
    \[ \iota\mathord\equiv~\mathcal{G}: \mathcal{G} \equiv \AgdaFunction{RSymGroup}~\mathcal{G} \]
  \item<+-> This lets us define:
    \begin{code}
      strictify :  (𝓖 : Group ℓ-zero)
                →  (P : Group ℓ-zero → Type)
                →  P (RSymGroup 𝓖)
                →  P 𝓖
      strictify 𝓖 P p = transport (sym (cong P (ι≡ 𝓖))) p
    \end{code}
  \end{itemize}
\end{frame}

\section{Further thoughts}
\begin{frame}
  \frametitle{Further thoughts}
  \pause{}
  \begin{center}
    Does this all work with categories instead of groups?
  \end{center}
\end{frame}
\begin{frame}
  \frametitle{Conclusion}
  \begin{itemize}
  \item For each group \(\mathcal{G}\) we can generate an isomorphic group \(\AgdaFunction{RSymGroup}~\mathcal{G}\).
  \item This group has nice definitional properties
  \item Univalence allows us to generate an equality between the two groups.
  \item This allows us to prove theorems about an arbitrary group by instead proving them on the strictified group.
  \item \small\url{https://alexarice.github.io/posts/sgtuf/Strict-Group-Theory-UF.html}
  \end{itemize}

\end{frame}

\end{document}
